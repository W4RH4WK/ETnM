% vim:set spell tw=79:

\documentclass[beamer]{uibk}
\title{Exploitation Techniques and Mitigations}
\subtitle{Dark Arts of Computer Science}
\author{Alex Hirsch \and Patrick Ober}
\date{2016-01-15}

\usepackage{svg}

\AtBeginSection[] {
    \begin{frame}{Outline}
        \tableofcontents[currentsection]
    \end{frame}
}

\begin{document}

\maketitle

% \begin{frame}{Outline}
%     \tableofcontents
% \end{frame}

\begin{frame}{Acknowledgement}
    \begin{columns}
        \begin{column}{0.5\textwidth}
            We use a lot from RPISEC, a university course about modern
            exploitation at Rensselaer Polytechnic Institute (2015), because
            \dots

            \begin{description}
                \item[of them:] They did a great job.
                \item[of you:] You will see familiar material.
                \item[of us:] We are lazy.
            \end{description}

            Check them out at \url{http://rpis.ec/} and
            \url{https://github.com/RPISEC/MBE}.
        \end{column}
        \begin{column}{0.5\textwidth}
            \includegraphics[height=0.7\textheight]{ripsec}
        \end{column}
    \end{columns}
\end{frame}

\section{Platform x86}

\begin{frame}{Why?}
    \begin{itemize}
        \item It's simpler, yet not overly simplified.
        \item Most techniques can be translated easily.
        \item Most material covers x86.
    \end{itemize}
\end{frame}

\begin{frame}{Registers}
    \begin{center}
        \includegraphics[width=\textwidth]{x86_registers}
    \end{center}
    \sidenote{- Wikipedia}
\end{frame}

\begin{frame}{Memory Management}
    \begin{columns}
        \begin{column}{0.45\textwidth}
            Real memory is managed by your OS kernel, a process sees only
            \textbf{virtual} memory.

            \medskip

            Memory is segmented (\textbf{pages}), which are handled by hardware
            (\textbf{memory management unit}) and software (kernel).

            \medskip

            \SI{4}{\kibi\byte} typical page size, addresses can be decomposed,
            page pointer + offset:

            \[\mathtt{0xA1B2C3D4} \implies \texttt{0xA1B2C000} + \mathtt{0x3D4}\]
        \end{column}
        \begin{column}{0.55\textwidth}
            \includegraphics[height=0.7\textheight]{pages}
        \end{column}
    \end{columns}
    \sidenote{- Unix Internals by Uresh Vahalia}
\end{frame}

\begin{frame}{Process' Memory}
    \begin{columns}
        \begin{column}{0.5\textwidth}
            \begin{center}
                \includegraphics[height=0.8\textheight]{mem_layout}
            \end{center}
        \end{column}
        \begin{column}{0.5\textwidth}
            You know some of this, other talks also focus on this.

            \bigskip

            We'll see:
            \begin{itemize}
                \item Pages have permissions \texttt{rwx} (DEP)
                \item Layout not always the same (ALSR)
                \item Lot's of pointers
            \end{itemize}
        \end{column}
    \end{columns}
\end{frame}

\begin{frame}{Calling Convention}
    \begin{columns}
        \begin{column}{0.5\textwidth}
            Defines:

            \begin{itemize}
                \item where to place arguments
                \item where to place return value
                \item where to place return address
                \item who prepares the stack
                \item who cleans afterwards\\
                    (caller vs.\ callee)
            \end{itemize}

            Depends on:

            \begin{itemize}
                \item your platform
                \item your toolchain (language)
                \item your settings (compiler flags)
            \end{itemize}

        \end{column}
        \begin{column}{0.5\textwidth}
            I know, Radu never told you.\dots

            \medskip

            \textbf{C Declaration:}

            \begin{itemize}
                \item arguments on stack (reverse order)\\
                    aligned to \SI{16}{\byte} boundary
                \item return via register (\texttt{EAX} / \texttt{ST0})
                \item return address on stack (old \texttt{IP})
                \item old base pointer (\texttt{BP})
            \end{itemize}
        \end{column}
    \end{columns}
\end{frame}

\begin{frame}{System Call \& Protection Rings}
    \begin{columns}
        \begin{column}{0.55\textwidth}
            \includegraphics[width=\textwidth]{x86_rings}
        \end{column}
        \begin{column}{0.45\textwidth}
            Your CPU can switch from a more privileged state to a less
            privileged one.

            \bigskip

            Kernel does not run always, process cannot do everything (enforced
            by hardware).

            \bigskip

            Process uses a \emph{System Call} (own instruction) to notify the
            kernel to take over (context switch).
        \end{column}
    \end{columns}
\end{frame}

\end{document}
